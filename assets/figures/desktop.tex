\documentclass{standalone}
\usepackage{ctex}
\usepackage{hyperref}
\usepackage{ulem}
\usepackage{graphicx}
\usepackage{makecell}
\title{desktop environments}
\begin{document}
\begin{tabular}{|c|c|c|c|}
  \hline
  Linux 桌面环境(DE)九宫格 &
  \makecell{\textbf{用户善良}                                \\功能多才是好桌面环境} &
  \makecell{\textbf{用户中立}                                \\功能够用才是好桌面环境} &
  \makecell{\textbf{用户邪恶}                                \\桌面环境是什么?能吃吗?}
  \\\hline
  \makecell{\textbf{语言保守}                                \\C 才是 Linux 软件开发的正统} &
  \makecell{\includegraphics[width=5em]{gnome.png}       \\\href{https://www.gnome.org/}{Gnome}\\\href{https://ubuntu.com/}{Ubuntu} 默认桌面\\我占用率大我先说话} &
  \makecell{\includegraphics[width=5em]{xfce-mirror.png} \\\href{https://www.xfce.org}{Xfce}\\\href{https://github.com/peng-zhihui/}{稚晖君}代言\\比什么功能,我们比内存占用} &
  \makecell{\includegraphics[width=5em]{i3.png}          \\\href{https://i3wm.org/}{i3}\\i3 大法好\\退 DE 保全家}
  \\\hline
  \makecell{\textbf{语言中立}                                \\OOP 才是桌面环境开发的王道} &
  \makecell{\includegraphics[width=5em]{kde.png}         \\\href{https://kde.org/}{KDE}\\\href{https://tysontan.com/}{钛山}认证, \href{https://archlinux.org/}{ArchLinux} \href{https://pkgstats.archlinux.de/fun/}{投票榜第一}\\是我,是我先,明明都是我先来的} &
  \makecell{\includegraphics[width=5em]{lxqt.png}        \\\href{https://lxqt-project.org/}{LXQt}\\不\href{https://github.com/PCMan/}{会写桌面环境的内科医生}\\不是好的软件工程师} &
  \makecell{\includegraphics[width=5em]{awesomeWM.png}   \\\href{https://awesomewm.org/}{awesome}\\如果我拿出 lua\\阁下又该如何应对呢}
  \\\hline
  \makecell{\textbf{语言混乱}                                \\邪教} &
  \makecell{\includegraphics[width=5em]{microsoft.png}   \\\href{https://learn.microsoft.com/en-us/windows/wsl/about}{WSL2}\\WSL2 也是 GNU/Linux \sout{的虚拟机}\\Windows 才是最好的桌面环境} &
  \makecell{\includegraphics[width=5em]{android.png}     \\\href{https://www.android.com/}{Android}\\Android 也是 Android/Linux\\Android 才是最好的桌面环境} &
  \makecell{\includegraphics[width=5em]{openwrt.png}     \\\href{https://openwrt.org/}{OpenWrt}\\OpenWrt 也是 \href{https://www.musl-libc.org/}{musl}/Linux\\都有命令行了还要什么自行车?}
  \\\hline
\end{tabular}
\end{document}
